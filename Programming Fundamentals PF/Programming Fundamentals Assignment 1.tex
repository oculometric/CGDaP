% Options for packages loaded elsewhere
\PassOptionsToPackage{unicode}{hyperref}
\PassOptionsToPackage{hyphens}{url}
%
\documentclass[
]{article}
\usepackage{amsmath,amssymb}
\usepackage{iftex}
\ifPDFTeX
  \usepackage[T1]{fontenc}
  \usepackage[utf8]{inputenc}
  \usepackage{textcomp} % provide euro and other symbols
\else % if luatex or xetex
  \usepackage{unicode-math} % this also loads fontspec
  \defaultfontfeatures{Scale=MatchLowercase}
  \defaultfontfeatures[\rmfamily]{Ligatures=TeX,Scale=1}
\fi
\usepackage{lmodern}
\ifPDFTeX\else
  % xetex/luatex font selection
\fi
% Use upquote if available, for straight quotes in verbatim environments
\IfFileExists{upquote.sty}{\usepackage{upquote}}{}
\IfFileExists{microtype.sty}{% use microtype if available
  \usepackage[]{microtype}
  \UseMicrotypeSet[protrusion]{basicmath} % disable protrusion for tt fonts
}{}
\makeatletter
\@ifundefined{KOMAClassName}{% if non-KOMA class
  \IfFileExists{parskip.sty}{%
    \usepackage{parskip}
  }{% else
    \setlength{\parindent}{0pt}
    \setlength{\parskip}{6pt plus 2pt minus 1pt}}
}{% if KOMA class
  \KOMAoptions{parskip=half}}
\makeatother
\usepackage{xcolor}
\usepackage[left=30mm,right=30mm]{geometry}
\setlength{\emergencystretch}{3em} % prevent overfull lines
\providecommand{\tightlist}{%
  \setlength{\itemsep}{0pt}\setlength{\parskip}{0pt}}
\setcounter{secnumdepth}{-\maxdimen} % remove section numbering
\ifLuaTeX
  \usepackage{selnolig}  % disable illegal ligatures
\fi
\IfFileExists{bookmark.sty}{\usepackage{bookmark}}{\usepackage{hyperref}}
\IfFileExists{xurl.sty}{\usepackage{xurl}}{} % add URL line breaks if available
\urlstyle{same}
\hypersetup{
  hidelinks,
  pdfcreator={LaTeX via pandoc}}

\author{}
\date{}

\begin{document}

\subsection{Task 1}\label{task-1}

To simplify and keep the code tidy, I wrote the ASCII-art as a
multi-line string, rather than \texttt{cout}-ing many individual lines.
Since backslash is treated as the indicator of an escape character, I
had to replace each backslash with a double backslash in order to
produce a single backslash in the output. I found this task easy, as I'm
already very familiar with C++ syntax. The ASCII art text was generated
using https://patorjk.com/software/taag

\subsection{Task 2}\label{task-2}

This task is mostly just making use of \texttt{cout} and \texttt{cin} as
well as the \texttt{\textless{}\textless{}} and
\texttt{\textgreater{}\textgreater{}} stream operators to assemble
multiple strings into a single console line. I made sure to use
\texttt{endl} rather than just \texttt{"\textbackslash{}n"} because it
reads more clearly and it ensures consistency across platforms
(\texttt{CRLF} vs \texttt{LF} vs \texttt{CR}). I used the standard
library function \texttt{int\ stoi(string)} to convert the age text
input from the user into an integer. I again found this task easy since
I'm already familiar with C++. I later came back and added better input
handling, using \texttt{getline} instead of just
\texttt{cin\ \textgreater{}\textgreater{}}, and wrapping \texttt{stoi}
conversions in \texttt{try\ \{\}\ catch\ \{\}} blocks.

\subsection{Task 3}\label{task-3}

This task is similar to the previous task in that it primarily involves
getting input from the user via \texttt{cin}, though this time I used
\texttt{float\ stof(string)} to get a floating point number from the
user's text input instead of \texttt{stoi}. I could have used the
\texttt{float\ powf(float,\ float)} function to square the number, but I
used \texttt{num*num} instead since this is faster in an instance like
this with a pre-determined integer power (in this case 2). Again I later
came back and improved input handling by wrapping \texttt{stof}
conversions in \texttt{try\ \{\}\ catch\ \{\}} blocks.

\subsection{Task 4}\label{task-4}

I split my box-making code off into its own function for modularity; it
takes two arguments: the text to put in the box, and the padding amount.
The length of the string plus padding and box edges is stored as a local
variable to simplify conditions in for loops later. The code then passes
over multiple for loops to do things like generating a whole line of
stars for the top, generate a line of spaces bounded by stars (and then
output that a number of times based on the padding amount), and output
the padding around the actual text. I used inline for loops for all of
these, since each loop only contains a single
\texttt{cout\ \textless{}\textless{}\ something} statement, which
considerably shortens the code and makes it easier to read. I generated
the padding line (i.e.~line of spaces with a star at each end) as a
string and then repeatedly outputted that string, rather than building
the padding line again each time, avoiding the use of nested for loops
and making the code more efficient. I also discovered over the course of
this challenge that \texttt{cin\ \textgreater{}\textgreater{}} supports
automatically converting its result to an integer if the destination
variable is an integer, which would slightly simplify some code in
previous challenges. However I still used \texttt{getline} for the text
itself so that the user can type multiple words without problems.

\subsection{Task 5}\label{task-5}

My code is split into three functions, one for fully capitalising a
string, one for full decapitalising it, and one for capitalising it on
only the first letter of a sentence. The code primarily makes use of the
standard library \texttt{char\ toupper(char)} and
\texttt{char\ tolower(char)} functions to capitalise and decapitalise
characters easily. The first two functions are simple \texttt{for} loops
where we loop over the input and add the capitalised/decapitalised
letter to the output buffer, before we return this buffer at the end.
The latter function is more complex:

\begin{itemize}
\tightlist
\item
  We keep track of the current state of the text, with a local variable
  telling the program whether it is waiting to capitalise the first
  letter of a new sentence
\item
  While this flag is set, we check to see if the current character is an
  alphabetical letter, by comparing it to capital and lowercase `A' and
  `Z' (since alphabetical letters in the ASCII character set are laid
  out, very sensibly, contiguously, though capital and lowercase are
  separated). Doing this with the
  \texttt{chr\ \textgreater{}=\ \textquotesingle{}A\textquotesingle{}\ \&\&\ chr\ \textless{}=\ \textquotesingle{}Z\textquotesingle{}}
  condition format is great because it makes it easy to see what is
  actually being checked
\item
  If the flag is set and the current character is an alphabetical
  letter, we must capitalise it, add it to the output buffer, and clear
  the end-of-sentence flag
\item
  If the flag is set but there is no alphabetical letter (e.g.~a space
  after a fullstop), we add the letter to the output buffer but don't
  clear the flag
\item
  Otherwise, we just add the character to the output buffer, checking to
  see if the character we're adding is a fullstop, exclamation mark, or
  question mark, and if it is then we set the end-of-sentence flag
\end{itemize}

The main problem I encountered was that
\texttt{cin\ \textgreater{}\textgreater{}\ text} only gets up to the
next whitespace character and puts it in \texttt{text}, meaning that
strings with spaces in (i.e.~most sentences written by humans) would
only get their first word read. This was fixed by using the
\texttt{void\ getline(istream\ \&,\ string)} function, which pulls an
entire line up until it reads a newline character into the provided
string variable.

\subsection{Task 6}\label{task-6}

This task was made a lot easier by the fact that the random function is
already written for us, so just calling it with appropriate arguments
was simple. However, since the function defined in main.h is not
documented, it required a little bit of experimentation to prove whether
or not the upper and lower bounds were inclusive (through testing I
found that they are both inclusive). We can use a simple
\texttt{while\ (true)\ \{\ \}} loop with a \texttt{break} statement when
the player guesses the number. Alternatively a conditional loop could be
used, but this setup has the advantage that we don't need to duplicate
code to prompt the user for input (i.e.~we don't need to have an
out-of-loop first prompt before the loop begins). Inside this
\texttt{while} loop we prompt the user for a guess with \texttt{cout}
and \texttt{cin}, check how far away from the secret number it is, break
if its equal, or otherwise give the user a hint as to how warm they are:

\begin{itemize}
\tightlist
\item
  We first calculate the absolute difference from the secret number to
  the user guess, making the following \texttt{if} statements much
  simpler
\item
  The next part was actually the trickiest part, since the description
  of the challenge talks about the player guessing \textbf{exactly 50
  away} for the `freezing' response for instance, and does not specify
  if the range of difference values covered by `freezing' should
  therefore be 50-100 or 35-50. After somewhat clarifying with a tutor,
  I treated the challenge description as meaning \textbf{50 or more
  away} from the secret number (and following the same method for the
  other ranges)
\item
  This did mean adjusting the `boiling' range to also cover being 1 away
  from the secret number; previously I had an extra condition for being
  this close named `super duper boiling'
\item
  I used a chain of inline \texttt{if...else\ if...else} statements to
  implement these ranges in order to keep the code concise and the
  conditions on the \texttt{if} statements simple, and to avoid nesting
\end{itemize}

Each iteration of the \texttt{while} loop also increments the guess
counter variable. When we escape the loop, the user is told what the
number was and how many guesses they took to find it. I later added
better input checking to this task, by way of using \texttt{getline} for
the user input and adding a \texttt{try\ \{\}\ catch\ \{\}} block around
the string-to-int conversion.

\subsection{Task 7}\label{task-7}

I made use of a \texttt{const\ vector\textless{}string\textgreater{}} to
store the character classes, since it's easy to get the length of this
data type. I wrote a utility function which outputs a description of my
character details struct. This could have been written as an overload of
the \texttt{\textless{}\textless{}} operator. My character details
struct contains two members, a name as a string, and a class ID as an
integer (where the integer indexes the character class vector). When
getting the user's player class, I used a simple
\texttt{while\ (true)\ \{\ \}} loop, which prompts the user with the
list of character classes each iteration, reads their input as an
integer, and breaks only if the user's input was within the valid range.
The user is also told if their input is valid or invalid, and which
class they picked if they pick a valid option. If the user doesn't enter
a number at all, a \texttt{try\ \{\}\ catch\ \{\}} block handles this
and prompts the user for input again. When getting the user's character
name I used \texttt{getline} in order to ensure the player could enter
multi-word names (like maybe, `Argoth the Destroyer', or whatever). I
initially had problems with this however, since \texttt{getline} would
return nothing (leading to a blank character name), skipping immediately
due to it detecting the trailing newline from the previous user input. I
resolved this by using \texttt{cin.ignore()} just before
\texttt{getline}, to tell it to ignore the trailing newline (as
suggested by answers on stack overflow). I used a pointer to my
character details struct, so that a character could be passed easily
into functions such as the \texttt{describe\_character} function, and
written back to by those functions potentially (I could imagine a
function which might modify the character to upgrade their stats).

\subsection{Task 8}\label{task-8}

My struct representing an inventory slot contains two fields: an item ID
\texttt{int}, which indexes the \texttt{item\_descriptions} vector
(where -1 represents an empty slot), and an item description
\texttt{string}. I again used a
\texttt{const\ vector\textless{}string\textgreater{}} to store the list
of item descriptions, as it's easily expandable and has a useful
\texttt{.size()} method for performing bounds checks. I also developed a
few utility functions:

\begin{itemize}
\tightlist
\item
  A function which just gets an item description from an ID (an
  \texttt{unsigned\ int}), and just contains a protective \texttt{if}
  statement to prevent indexing the item descriptions \texttt{vector}
  out of range
\item
  A \texttt{split\_string} function which takes in a string and a char,
  and will use the \texttt{string.find} and \texttt{string.substr}
  methods to split the input into a
  \texttt{vector\textless{}string\textgreater{}} at the delimiting
  character. This method is only used once but is very important to the
  command processing code and makes sense to have it abstracted as a
  separate function for clarity and modularity
\end{itemize}

The `inventory' itself is managed by a class. This allows us to hide the
actual backing data for the inventory, a
\texttt{vector\textless{}inventory\_slot\textgreater{}}, which can be
protected by only allowing access through methods (the backing data
member is made private). These methods allow for setting the contents of
slots to particular item IDs, clearing slots, getting the contents of an
inventory slot, setting the size of the inventory, and getting the size.
Each of these methods has the appropriate validation, including checking
for indexing the inventory out of range The user is prompted for a
positive integer to initialise the inventory size, in a loop which
repeats until a valid input is received. The only new thing to note here
is handling exceptions from the \texttt{stoi} function using a
\texttt{try...catch} block. If this function throws an error, we should
simply prompt the user for another input. After initialising the
inventory, we start a main program loop in which commands are processed,
this is a \texttt{while} loop predicated on the
\texttt{continue\_mainloop} variable which allows us to break out from
inside nested loops. When getting the user's command input, I had some
problems again with \texttt{getline} being tripped up by a trailing
newline, but having multiple inputs previously seemed to cause
unpredictable behaviour. I solved this by repeatedly calling
\texttt{getline} until it manages to get some input (i.e.~a
\texttt{string} of length \textgreater{} 0) without actually
re-prompting the user. After the string is split at the space character,
the only thing left is a lot of \texttt{if} statements to perform
validation and branch depending upon the command entered, and the
arguments passed to the command. Depending on the command, we must check
the number of arguments provided (by looking at the number of elements
in the output of \texttt{split\_command}), and check that the numerical
values entered are valid and in range for what they're indexing
(e.g.~the inventory or the list of items). Whenever an error condition
is encountered, such as a `set' command not having enough arguments, the
\texttt{continue} keyword is used to skip back to the top of the loop
and ask the user for a fresh command. I tested my input validation with
a variety of different inputs, (valid, invalid, and edge case), so I'm
confident that it would catch all invalid inputs. Although I didn't have
any real problems with this challenge besides the \texttt{getline}
shenanigans, it was a more interesting challenge than previous ones. I
later added more input handling to prevent exceptions, specifically
adding \texttt{try\ \{\}\ catch\ \{\}} blocks around calls to
\texttt{stoi} and re-prompting the user for input if an exception
occurs.

\subsection{Task 9}\label{task-9}

The \texttt{Vector2} class itself is very simple, with
\texttt{float\ x,\ y} variables. However, for some OOP practice I
embedded \texttt{magnitude} and \texttt{distance} methods in the class,
as well as a constructor which takes two \texttt{float}s for x and y.
This \texttt{magnitude} method does exactly what it sounds like, and its
implementation uses the same approach as earlier of \texttt{x\ *\ x}
instead of \texttt{powf(x,\ 2)} for a slight speed boost, and the
\texttt{sqrt} function to complete Euclid's formula. The distance method
takes another \texttt{Vector2} as input and subtracts the self
\texttt{Vector2} from it, before returning the magnitude of the
resulting \texttt{Vector2}. In order to do this subtraction, I wrote a
set of operator overloads accepting \texttt{Vector2} types: adding,
subtracting, multiplying and dividing two \texttt{Vector2}s, and also
multiplying and dividing a \texttt{Vector2} by a float. I also wrote a
\texttt{\textless{}\textless{}} stream operator overload so that I could
easily output my \texttt{Vector2} type to the console. However, these
operators created a problem: they need the \texttt{Vector2} class to be
declared, including x and y fields, meanwhile the class methods, mainly
the \texttt{distance} method require the operators to already be
declared (subtraction specifically), leading to a dependency loop. I
solved this by declaring the class as a stub, then defining the
operators, then finally implementing the methods from the
\texttt{Vector2} class. Ideally these implementations would be done in a
dedicated file (to form a pair of `vector2.h' and `vector2.cpp'). As
instructed by the challenge description, I also wrote a
\texttt{GetDistanceBetweenPoints} function which simply calls the
\texttt{Vector2::distance} method and returns the result. The rest of
the code back in main is just constructing a pair of \texttt{Vector2}s,
calling \texttt{GetDistanceBetweenPoints} and outputting the result.
This was an interesting challenge! I've had practice writing operators
and mathematical classes before so this was a reminder for me.

\subsection{Task 10}\label{task-10}

I came up with two \texttt{struct}s one for the marking information of
each assessment, with the appropriate members based on reading the CRG.
I had the idea of having the structs initialised with values equal to
the maximum mark for those grades, meaning that to find the maximum
marks for things (e.g.~for finding percentage marks), I can just
instantiate a new struct and read its default values. I wrote a few
utility functions:

\begin{itemize}
\tightlist
\item
  One for adding up all the marks in the different challenges of the
  first criterion in assessment 1
\item
  One for getting the total, weighted score for assessment 1, using the
  70-30 split as described by the CRG
\item
  One for getting the total, weighted score for assessment 2, using the
  60-20-20 split as described by the CRG
\item
  One which abstracts away the process of getting a validated integer
  input from the user. This gets used \textbf{a lot} in \texttt{main},
  and since we want to do a little \texttt{while} loop to repeat until
  we get valid input, it makes sense to have this as a self-contained
  function. It supports having custom upper and lower bound limits and a
  specific prompt for the user
\item
  One which returns a description for which CRG band a percentage score
  falls into, based on the ranges defined in the CRG
\end{itemize}

The rest of the program is lots of getting input from the user to
populate the structs \texttt{a1g} and \texttt{a2g}. Once all this is
done, we can calculate extra information like the total assessment 1
criterion 1 score, the total percentage score of assessment 1, the total
percentage score of assessment 2, and the CRG grade bands for both of
these and the individual criteria within. We can then output all of this
information, nicely formatted, to the user. Again, I tested this
thoroughly with various correct inputs (as well as checking the error
checking with invalid inputs) and the output matched what was expected
from what is written in the CRG.

\end{document}
