% Options for packages loaded elsewhere
\PassOptionsToPackage{unicode}{hyperref}
\PassOptionsToPackage{hyphens}{url}
%
\documentclass[
]{article}
\usepackage{amsmath,amssymb}
\usepackage{iftex}
\ifPDFTeX
  \usepackage[T1]{fontenc}
  \usepackage[utf8]{inputenc}
  \usepackage{textcomp} % provide euro and other symbols
\else % if luatex or xetex
  \usepackage{unicode-math} % this also loads fontspec
  \defaultfontfeatures{Scale=MatchLowercase}
  \defaultfontfeatures[\rmfamily]{Ligatures=TeX,Scale=1}
\fi
\usepackage{lmodern}
\ifPDFTeX\else
  % xetex/luatex font selection
\fi
% Use upquote if available, for straight quotes in verbatim environments
\IfFileExists{upquote.sty}{\usepackage{upquote}}{}
\IfFileExists{microtype.sty}{% use microtype if available
  \usepackage[]{microtype}
  \UseMicrotypeSet[protrusion]{basicmath} % disable protrusion for tt fonts
}{}
\makeatletter
\@ifundefined{KOMAClassName}{% if non-KOMA class
  \IfFileExists{parskip.sty}{%
    \usepackage{parskip}
  }{% else
    \setlength{\parindent}{0pt}
    \setlength{\parskip}{6pt plus 2pt minus 1pt}}
}{% if KOMA class
  \KOMAoptions{parskip=half}}
\makeatother
\usepackage{xcolor}
\usepackage[left=30mm,right=30mm]{geometry}
\usepackage{longtable,booktabs,array}
\usepackage{calc} % for calculating minipage widths
% Correct order of tables after \paragraph or \subparagraph
\usepackage{etoolbox}
\makeatletter
\patchcmd\longtable{\par}{\if@noskipsec\mbox{}\fi\par}{}{}
\makeatother
% Allow footnotes in longtable head/foot
\IfFileExists{footnotehyper.sty}{\usepackage{footnotehyper}}{\usepackage{footnote}}
\makesavenoteenv{longtable}
\usepackage{graphicx}
\makeatletter
\def\maxwidth{\ifdim\Gin@nat@width>\linewidth\linewidth\else\Gin@nat@width\fi}
\def\maxheight{\ifdim\Gin@nat@height>\textheight\textheight\else\Gin@nat@height\fi}
\makeatother
% Scale images if necessary, so that they will not overflow the page
% margins by default, and it is still possible to overwrite the defaults
% using explicit options in \includegraphics[width, height, ...]{}
\setkeys{Gin}{width=\maxwidth,height=\maxheight,keepaspectratio}
% Set default figure placement to htbp
\makeatletter
\def\fps@figure{htbp}
\makeatother
\setlength{\emergencystretch}{3em} % prevent overfull lines
\providecommand{\tightlist}{%
  \setlength{\itemsep}{0pt}\setlength{\parskip}{0pt}}
\setcounter{secnumdepth}{-\maxdimen} % remove section numbering
\ifLuaTeX
  \usepackage{selnolig}  % disable illegal ligatures
\fi
\IfFileExists{bookmark.sty}{\usepackage{bookmark}}{\usepackage{hyperref}}
\IfFileExists{xurl.sty}{\usepackage{xurl}}{} % add URL line breaks if available
\urlstyle{same}
\hypersetup{
  pdftitle={Computational Methods Assignment},
  hidelinks,
  pdfcreator={LaTeX via pandoc}}

\title{Computational Methods Assignment}
\author{}
\date{}

\begin{document}
\maketitle

\subsection{Task 1}\label{task-1}

\includegraphics{blob:app://obsidian.md/eefe82e8-cc60-4048-b00e-84b307dd68a2}

\hfill\break
I decided to write a program to brute-force this, rather than do it by
hand. Here is the pseudocode:

\begin{verbatim}
Let adjacency_matrix = {{0,10,15,12,20}, {10,0,12,25,14}, {15,12,0,16,28}, {12,25,16,0,17}, {20,14,28,17,0}}
Let cargo_pickup_weights = [20,40,70,10,30]

Function get_distance(a, b):
    Return adjacency_matrix[a][b]
End Function

Function calculate_fuel_cost(a, b, c, d, e):
    Let total_fuel = 0
    Let total_weight = 0

    total_weight = total_weight + cargo_pickup_weights[a]
    total_fuel = total_fuel + get_distance(a,b)*total_weight

    total_weight = total_weight + cargo_pickup_weights[b]
    total_fuel = total_fuel + get_distance(b,c)*total_weight

    total_weight = total_weight + cargo_pickup_weights[c]
    total_fuel = total_fuel + get_distance(c,d)*total_weight

    total_weight = total_weight + cargo_pickup_weights[d]
    total_fuel = total_fuel + get_distance(d,e)*total_weight

    total_weight = total_weight + cargo_pickup_weights[e]

    Return total_fuel*25
End Function

Let all_possible_sequences = []

Let a = 0
While a < 5:
    Let sequence = [a]

    Let b = 0
    While b < 5:
        If sequence Contains b:
            Continue
        End If
        Append b to sequence

        Let c = 0
        While c < 5:
            If sequence Contains c:
                Continue
            End If
            Append c to sequence

            Let d = 0
            While d < 5:
                If sequence Contains d:
                    Continue
                End If
                Append d to sequence

                Let e = 0
                While e < 5:
                    If sequence Contains e:
                        Continue
                    End If
                    Append e to sequence

                    Append sequence to all_possible_sequences
                    sequence = [a,b,c,d]
                    Increment e
                End While
                sequence = [a,b,c]
                Increment d
            End While
            sequence = [a,b]
            Increment c
        End While
        sequence = [a]
        Increment b
    End While

    Increment a
End While

Open "brute_force.csv" as file
Let index = 0
While index < Length of all_possible_sequences:
    Let seq = all_possible_sequences[index]
    Output seq[0] to file
    Output "," to file
    Output seq[1] to file
    Output "," to file
    Output seq[2] to file
    Output "," to file
    Output seq[3] to file
    Output "," to file
    Output seq[4] to file
    Output "," to file
    Output calculate_fuel_cost(seq[0], seq[1], seq[2], seq[3], seq[4]) to file
    Output "\n" to file
End While

Close file
Copy
\end{verbatim}

And the equivalent python code:

\begin{verbatim}
alpha = 0
beta = 1
gamma = 2
delta = 3
epsilon = 4

adjacency_matrix = [[0,10,15,12,20], [10,0,12,25,14], [15,12,0,16,28], [12,25,16,0,17], [20,14,28,17,0]]

cargo_pickup_weights = [20,40,70,10,30]

def get_distance(a, b):
    return adjacency_matrix[a][b]

def calculate_fuel_cost(a, b, c, d, e):
    total_fuel = 0
    total_weight = 0
    
    total_weight += cargo_pickup_weights[a]
    total_fuel += get_distance(a,b)*total_weight

    total_weight += cargo_pickup_weights[b]
    total_fuel += get_distance(b,c)*total_weight

    total_weight += cargo_pickup_weights[c]
    total_fuel += get_distance(c,d)*total_weight

    total_weight += cargo_pickup_weights[d]
    total_fuel += get_distance(d,e)*total_weight

    total_weight += cargo_pickup_weights[e]

    return total_fuel*25


all_possible_sequences = []

for a in range(5):
    sequence = [a]
    for b in range(5):
        if (b in sequence): continue
        sequence.append(b)
        for c in range(5):
            if (c in sequence): continue
            sequence.append(c)
            for d in range(5):
                if (d in sequence): continue
                sequence.append(d)
                for e in range(5):
                    if (e in sequence): continue
                    sequence.append(e)
                    all_possible_sequences.append(sequence)
                    sequence = [sequence[0], sequence[1], sequence[2], sequence[3]]
                sequence = [sequence[0], sequence[1], sequence[2]]
            sequence = [sequence[0], sequence[1]]
        sequence = [sequence[0]]
                    
csv_data = ""
for seq in all_possible_sequences:
    csv_data += str(seq[0]) + ","
    csv_data += str(seq[1]) + ","
    csv_data += str(seq[2]) + ","
    csv_data += str(seq[3]) + ","
    csv_data += str(seq[4]) + ","
    csv_data += str(calculate_fuel_cost(seq[0], seql[1], seq[2], seq[3], seq[4])) + "\n"

print("generated " + str(len(all_possible_sequences)) + " sequences")

file = open("brute_force.csv", "w")

file.write(csv_data)
file.close()
Copy
\end{verbatim}

Here is the CSV file which is produced by the python program, and a more
formatted Excel conversion:\\
brute\_force.csv\\
../brute\_force.xlsx

Reading from these we can see that the cheapest route is:\\
3 0 4 1 2 = Delta -\textgreater{} Alpha -\textgreater{} Epsilon
-\textgreater{} Beta -\textgreater{} Gamma\\
Costs 69000 intergalactic currency

This approach isn\textquotesingle t a good way to find the shortest path
since it requires checking the cost of an enormous and rapidly
increasing search space. Specifically {} possible routes, {} being the
number of planets. Factorial time, {} is a very very bad time complexity
and building a route with many destinations would take a very long time.
In order to evaluate the cost of each route, we also have to traverse
the whole list of planets representing the route, which is length {}, so
the real time complexity is {}.

\subsection{Task 2}\label{task-2}

\begin{longtable}[]{@{}llllllllllllllll@{}}
\toprule\noalign{}
0 & 1 & 2 & 3 & 4 & 5 & 6 & 7 & 8 & 9 & start & end & i & j & pivot
value & notes \\
\midrule\noalign{}
\endhead
\bottomrule\noalign{}
\endlastfoot
10 & 15 & 12 & 12 & 25 & 16 & 20 & 14 & 28 & 17 & 0 & 9 & -1 & 10 & 25 &
partition the whole list \\
& & & & & & & & & & & & 0 & 10 & & \\
& & & & & & & & & & & & 1 & 10 & & \\
& & & & & & & & & & & & 2 & 10 & & \\
& & & & & & & & & & & & 3 & 10 & & \\
& & & & & & & & & & & & 4 & 10 & & \\
& & & & & & & & & & & & 4 & 9 & & \\
10 & 15 & 12 & 12 & 17 & 16 & 20 & 14 & 28 & 25 & & & & & & swap 4 and
9 \\
& & & & & & & & & & & & 5 & 9 & & \\
& & & & & & & & & & & & 6 & 9 & & \\
& & & & & & & & & & & & 7 & 9 & & \\
& & & & & & & & & & & & 8 & 9 & & \\
& & & & & & & & & & & & 8 & 8 & & \\
& & & & & & & & & & & & 8 & 7 & & partitioning finished \\
10 & 15 & 12 & 12 & 17 & 16 & 20 & 14 & & & 0 & 7 & -1 & 8 & 12 &
subsort first half \\
& & & & & & & & & & & & 0 & 8 & & \\
& & & & & & & & & & & & 1 & 8 & & \\
& & & & & & & & & & & & 1 & 7 & & \\
& & & & & & & & & & & & 1 & 6 & & \\
& & & & & & & & & & & & 1 & 5 & & \\
& & & & & & & & & & & & 1 & 4 & & \\
& & & & & & & & & & & & 1 & 3 & & \\
10 & 12 & 12 & 15 & 17 & 16 & 20 & 14 & & & & & & & & swap 1 and 3 \\
& & & & & & & & & & & & 2 & 3 & & \\
& & & & & & & & & & & & 2 & 2 & & \\
& & & & & & & & & & & & 2 & 1 & & partitioning finished \\
10 & 12 & & & & & & & & & 0 & 1 & -1 & 2 & 10 & subsort first half of
first half \\
& & & & & & & & & & & & 0 & 2 & & \\
& & & & & & & & & & & & 0 & 1 & & \\
& & & & & & & & & & & & 0 & 0 & & subsort done \\
& & 12 & 15 & 17 & 16 & 20 & 14 & & & 2 & 7 & 1 & 8 & 17 & subsort
second half of first half \\
& & & & & & & & & & & & 2 & 8 & & \\
& & & & & & & & & & & & 3 & 8 & & \\
& & & & & & & & & & & & 4 & 8 & & \\
& & & & & & & & & & & & 4 & 7 & & \\
& & 12 & 15 & 14 & 16 & 20 & 17 & & & & & & & & swap 4 and 7 \\
& & & & & & & & & & & & 5 & 7 & & \\
& & & & & & & & & & & & 6 & 7 & & \\
& & & & & & & & & & & & 6 & 6 & & \\
& & & & & & & & & & & & 6 & 5 & & partitioning finished \\
& & 12 & 15 & 14 & 16 & & & & & 2 & 5 & 1 & 6 & 15 & subsort first half
of second half of first half \\
& & & & & & & & & & & & 2 & 6 & & \\
& & & & & & & & & & & & 3 & 6 & & \\
& & & & & & & & & & & & 3 & 5 & & \\
& & & & & & & & & & & & 3 & 4 & & \\
& & 12 & 14 & 15 & 16 & & & & & & & & & & swap 3 and 4 \\
& & & & & & & & & & & & 4 & 4 & & \\
& & & & & & & & & & & & 4 & 3 & & partitioning finished \\
& & 12 & 14 & & & & & & & 2 & 3 & 1 & 4 & 12 & subsort first half of
first half of second half of first half \\
& & & & & & & & & & & & 2 & 4 & & \\
& & & & & & & & & & & & 2 & 3 & & \\
& & & & & & & & & & & & 2 & 2 & & subsort done \\
& & & & 15 & 16 & & & & & 4 & 5 & 3 & 6 & 15 & subsort second half of
first half of second half of first half \\
& & & & & & & & & & & & 4 & 6 & & \\
& & & & & & & & & & & & 4 & 5 & & \\
& & & & & & & & & & & & 4 & 4 & & subsort done \\
& & & & & & 20 & 17 & & & 6 & 7 & 5 & 8 & 20 & subsort second half of
second half of first half \\
& & & & & & & & & & & & 6 & 8 & & \\
& & & & & & & & & & & & 6 & 7 & & \\
& & & & & & 17 & 20 & & & & & & & & swap 6 and 7 \\
& & & & & & & & & & & & 7 & 7 & & \\
& & & & & & & & & & & & 7 & 6 & & subsort done \\
& & & & & & & & 28 & 25 & 8 & 9 & 7 & 10 & 28 & subsort second half \\
& & & & & & & & & & & & 8 & 10 & & \\
& & & & & & & & & & & & 8 & 9 & & \\
& & & & & & & & & & & & & & & swap 8 and 9 \\
& & & & & & & & 25 & 28 & & & & & & \\
& & & & & & & & & & & & 9 & 9 & & \\
& & & & & & & & & & & & 9 & 8 & & subsort done \\
10 & 12 & 12 & 14 & 15 & 16 & 17 & 20 & 25 & 28 & & & & & & sort done \\
\end{longtable}

This trace table represents a quicksort, and below is the pseudocode for
it.

\begin{verbatim}
Procedure swap(array, first, second) Being:
    Let temp = array[first]
    array[first] = array[second]
    array[second] = temp
End Procedure

Function partition_array(array, start, end) Begin:
    // Place the pivot in the middle, this tends to have better performance
    Let pivot_index = ((end - start)/2 Rounded Down) + start
    Let pivot = array[pivot_index]
    
    // Initialise pointers
    Let i = start - 1
    Let j = end + 1
    
    While True:
        // Increment i then break if the targeted element is swappable
        While True:
            Increment i
            If array[i] >= pivot:
                Break
            End If
        End While
        
        // Decrement j then break if the targeted element is swappable
        While True:
            Decrement j
            If array[j] <= pivot:
                Break
            End If
        End While
        
        // Return the partition point if i and j meet/cross, otherwise swap their values
        If i >= j:
            Return j
        Else:
            swap(array, i, j)
        End If
    End While
End Function

Procedure quick_sort(array, start, end) Begin:
    // Return if there is nothing to sort
    If start >= end:
        Return
    End If
    
    // Perform first sorting pass over current whole array
    Let split_index = partition_array(array, start, end)
    
    // Perform subsorts on partitioned arrays
    quick_sort(array, start, split_index)
    quick_sort(array, split_index + 1, end)
End Procedure
Copy
\end{verbatim}

This is an implementation of the quicksort algorithm, using
Hoare\textquotesingle s pivot choice and pair-of-pointers method. It
makes use of recursive quicksort calls to sort a list by swapping items
so that they effectively end up grouped (in each sublist) in groups of
larger and smaller items; these sublists can then be sorted using the
same method, until there is only one item in each sublist (this is the
base case for the recursion). This is an example of a divide-and-conquer
approach, as the subsequent quicksorts can be parallelised, since they
are independent from one another. Quicksort, depending on implementation
(particularly choice of pivot) as well as how sorted data already is,
usually has worst-case complexity {}. However with
Hoare\textquotesingle s partitioning scheme using the middle-pivot (as
opposed to pivoting at the start or end value) tends to have worst-case
complexity of {}.

\subsection{Task 3}\label{task-3}

\includegraphics{blob:app://obsidian.md/01dd2464-c88a-4e59-8a1d-6f364e1e8a77}

\hfill\break
Bearing in mind a greedy strategy chooses the best option in the short
term and does not look ahead, I experimented with two different
techniques: first, to traverse the graph choosing to move along the
\textbf{cheapest weighted edge} (excluding any which lead to already
visited nodes) at every node; second, to traverse along the edge to the
\textbf{lowest cargo mass} (using mass here to be distinct from edge
weights) \textbf{adjacent unvisited} node.\\
The second of these produced a resulting route (starting at delta, since
it has the lowest cargo mass to collect) of
\texttt{delta\ -\textgreater{}\ alpha\ -\textgreater{}\ epsilon\ -\textgreater{}\ beta\ -\textgreater{}\ gamma},
costing \textbf{69000} intergalactic Vbucks, which happens to also be
the optimal path found by the brute-force method.\\
The first approach by contrast, starting at the same place, ended up
choosing a
\texttt{delta\ -\textgreater{}\ gamma\ -\textgreater{}\ beta\ -\textgreater{}\ alpha\ -\textgreater{}\ epsilon}
route, which cost almost double the other method at \textbf{126500}
intergalactic Vbucks.\\
It makes sense that a mass-focused route is better, since mass
accumulates during the graph traversal, whereas the edge weightings
(distances between planets) do not accumulate in the same way.

I originally wrote an implementation which used a bubble sort to
pre-sort the planet cargo masses, followed by lots of lookups of planet
data, as I was keeping the graph data (edge weights, planet cargo
masses, etc) in a set of lists. This resulted in a rather unclear {}
algorithm. I decided to come back to this exercise during a seminar,
since I wasn\textquotesingle t happy with {} complexity, and the general
unclearness of the algorithm. I rewrote the greedy strategy using the
same basic algorithm, but with a better implementation. now planets are
stored as \textbf{structures}, containing all the information about them
and how they connect, meaning the amount of look-ups in lists is
significantly reduced.

I found that, surprisingly, when constructing the list of connections
that one node (i.e. planet) has with other nodes, it\textquotesingle s
actually \emph{better} to not bother sorting the list by cargo mass
(i.e. to reduce searching later). This is because the later code not
only needs to find the next lowest cargo mass planet, it needs to
\emph{find one which has not already been visited}, meaning we end up
doing a linear search through the connected planets regardless. This
means choosing between either an {} sorting pass with an {} traversal
pass, or an {} preparation pass with an {} traversal pass, the latter of
which is clearly better. The pseudocode and C++ implementation for the
improved method are below.

\begin{verbatim}
Let NUM_PLANETS = 5
Let FUEL_COST = 25

// structure holding data about a planet (i.e. a node)
Structure planet
    Let name = ""
    Let index = 0
    Let cargo_mass = 0
    Let links = {}
End Structure

// starting data about the planets
Let node_names = { "alpha", "beta", "gamma", "delta", "epsilon" }
Let cargo_masses = { 20,40,70,10,30 }
Let adjacency_matrix = { {0,10,15,12,20}, {10,0,12,25,14}, {15,12,0,16,28}, {12,25,16,0,17}, {20,14,28,17,0} }

// initialise the nodes in the graph
Let planets = {}
Let i = 0
While i < NUM_PLANETS
    Let p = Create planet
    name Of p = node_names[i]
    cargo_mass Of p = cargo_masses[i]
    index Of p = i
    Append p To planets
    
    Increment i
End While

Let p_minimum = planets[0]

// setup links between nodes
For p_origin In planets

    // update the lowest cargo mass planet
    // since we want to start at this planet
    // and this saves using a second loop
    If cargo_mass Of p_origin < cargo_mass Of p_minimum
        p_minimum = p_origin
    End If

    // add links from this node to all other nodes
    // but not itself
    Let i = 0
    While i < NUM_PLANETS
        If planets[i] Not p_origin
            Let distance = adjacency_matrix[i][index Of p_origin]
            Append {planets[i], distance} To links Of p_origin
        End If
        Increment i
    End While
End For

// traverse the graph, keeping track of which planets
// have been visited, and which haven't
Let visited = {}
Fill visited With False NUM_PLANETS Times

Let spaceship_mass = 0
Let fuel_cost = 0
Let sequence = ""

Let p_current = p_minimum
Let p_next Be Empty

Loop Forever

    // find the unvisited planet from the current
    // with the lowest cargo mass, via linear search
    Let minimum_mass = Infinity
    Let d_min_mass = -1
    Let p_min_mass Be Empty

    For l_candidate In links Of p_current
        Let p_candidate = l_candidate[0]
        If visited[index Of p_candidate] = False
            If cargo_mass Of p_candidate < minimum_mass
                minimum_mass = cargo_mass Of p_candidate
                p_min_mass = p_candidate
                d_min_mass = l_candidate[1]
            End If
        End If
    End For

    // if there were no unvisited planets
    // other than the current one, break from the loop
    If p_min_mass Is Empty
        Break Loop
    End If

    // update the next planet we want to visit
    // this will be the one with the next lowest cargo mass
    p_next = p_min_mass
    d_next = d_min_mass

    // add on the calculate fuel cost for the journey between
    // the current planet and the next
    spaceship_mass = spaceship_mass + cargo_mass Of p_current
    fuel_cost = fuel_cost + spaceship_mass * d_next * FUEL_COST

    // update the sequence string
    sequence = sequence + name Of p_current + " -> "

    // mark this planet as visited
    visited[index Of p_current] = true

    // move onto the next planet
    p_current = p_next
End Loop

// finalise and output the result
sequence = sequence + name Of p_current

Output "Found sequence: " + sequence
Output "Costing: " + fuel_cost
Copy
\end{verbatim}

\begin{verbatim}
#include <string>
#include <vector>
#include <iostream>

#define NUM_PLANETS 5
#define FUEL_COST 25

using namespace std;

// data about a planet
struct planet
{
    string name;
    int index;
    int cargo_mass;
    vector<pair<planet*, int>> links;
};

int main()
{
    // starting data about planets
    string node_names[NUM_PLANETS] = { "alpha", "beta", "gamma", "delta", "epsilon" };
    int cargo_masses[NUM_PLANETS] = { 20,40,70,10,30 };
    int adjacency_matrix[NUM_PLANETS][NUM_PLANETS] = { {0,10,15,12,20}, {10,0,12,25,14}, {15,12,0,16,28}, {12,25,16,0,17}, {20,14,28,17,0} };

    // create nodes
    vector<planet*> planets;
    for (int i = 0; i < NUM_PLANETS; i++)
    {
        planet* p = new planet();
        p->name = node_names[i];
        p->cargo_mass = cargo_masses[i];
        p->index = i;

        planets.push_back(p);
    }

    planet* p_minimum = planets[0];

    // setup links between nodes
    for (planet* p_origin : planets)
    {
        // update lowest cargo planet while we're here
        // saves having another loop
        if (p_origin->cargo_mass < p_minimum->cargo_mass)
        {
            p_minimum = p_origin;
        }

        // add links to other nodes
        // not sorted, since sorting them would
        // actually take more time (O(n^2) inside an n-loop)
        for (int i = 0; i < NUM_PLANETS; i++)
        {
            if (planets[i] == p_origin) continue;

            p_origin->links.push_back(pair<planet*, int>(planets[i], adjacency_matrix[i][p_origin->index]));
        }
    }

    // traverse, keep list of visited
    bool visited[NUM_PLANETS] = { false };

    int spaceship_mass = 0;
    int fuel_cost = 0;
    string sequence = "";

    planet* p_current = p_minimum;
    planet* p_next = NULL;
    int d_next = 0;

    while (true)
    {
        // find the unvisited planet with the lowest cargo mass
        int minimum_mass = INT_MAX;
        int d_min_mass = -1;
        planet* p_min_mass = NULL;
        for (pair<planet*, int> l_candidate : p_current->links)
        {
            planet* p_candidate = l_candidate.first;
            if (visited[p_candidate->index]) continue;
            if (p_candidate->cargo_mass < minimum_mass)
            {
                minimum_mass = p_candidate->cargo_mass;
                p_min_mass = p_candidate;
                d_min_mass = l_candidate.second;
            }
        }

        // if there were no unvisited planets,
        // other than the current one, break out
        if (p_min_mass == NULL) break;

        // set the planet we intend to visit
        // next (the one with the lowest cargo mass)
        p_next = p_min_mass;
        d_next = d_min_mass;

        // add on the calculated fuel cost
        spaceship_mass += p_current->cargo_mass;
        fuel_cost += spaceship_mass * d_next * FUEL_COST;

        // update the sequence string
        sequence += p_current->name;
        sequence += " -> ";

        // mark it as visited
        visited[p_current->index] = true;

        // move onto the next
        p_current = p_next;
    }

    // output the result
    sequence += p_current->name;

    cout << "Found sequence: " << sequence << endl;
    cout << "Costing: " << fuel_cost << endl;

    return 0;
}
Copy
\end{verbatim}

Output:

\begin{quote}
Found sequence delta -\textgreater{} alpha -\textgreater{} epsilon
-\textgreater{} beta -\textgreater{} gamma\\
Costing: 69000
\end{quote}

The output from the second version of the algorithm is, aside from
cosmetic modification, exactly the same as from the earlier version. The
resulting algorithmic complexity of this solution is {}, due to the two
occurrences of traversing lists of length {}, {} times over.

The only further optimisation I think could be made would be using a
system of sorted lookup tables for planet data and cargo masses to
eliminate the need for searching for the next lowest \emph{unvisited}
cargo mass planet, which we just step through sequentially, removing the
need for linear searching at each iteration. This might reduce
complexity to {} in the best case if sorted with quicksort.

\subsection{Task 4}\label{task-4}

../alpha.csv\\
../beta.csv\\
../delta.csv\\
../epsilon.csv\\
../gamma.csv

I wrote a short C++ program to produce these tables, in other words I
not only followed the dynamic programming approach but also implemented
the algorithm. The raw exported CSV files are linked above, and then the
assembled and formatted Excel spreadsheet is below:

../dynamic\_programming.xlsx

My code primarily makes use of a \textbf{tree structure} which
represents the data which is eventually placed in the table, but which
is \textbf{more compact and easier to traverse}. I made use of a
\texttt{std::queue} to keep track of the next block of possible
sequences to test, and a \texttt{std::map} to keep track of the cheapest
version of similar routes (used for carrying forward only the better
routes). My tree structure makes use of \textbf{pointers} to other nodes
allocated on the heap. Code is below:

\begin{verbatim}
#include <map>
#include <string>
#include <queue>
#include <iostream>
#include <fstream>

// allows for much easier debugging
#define NODE_ZERO 65

using namespace std;

// only supports up to 255 nodes, since each node reference is only a single byte/char
#define NUM_NODES 5

// data describing the network
const int adjacency[NUM_NODES][NUM_NODES] = { {  0, 10, 15, 12, 20 },
                                                { 10,  0, 12, 25, 14 },
                                                { 15, 12,  0, 16, 28 },
                                                { 12, 25, 16,  0, 17 },
                                                { 20, 14, 28, 17,  0 }
};

const int weight[NUM_NODES] = { 20, 40, 70, 10, 30 };
const string names[NUM_NODES] = { "alpha", "beta", "gamma", "delta", "epsilon" };

// struct containing information about a node in the tree
struct cost_tree_node
{
    int cumulative_cost = 0;
    int cumulative_weight = 0;
    string planets_sequence = "";
    unsigned char last_planet = 0;
    cost_tree_node** children = NULL;
    cost_tree_node* parent = NULL;
};

// sort a string sequence alphabetically, but excluding the first and last characters
string sort_sequence(string seq)
{
    if (seq.length() <= 3) return seq;

    string to_sort = seq;

    bool changed = true;
    while (changed)
    {
        changed = false;
        for (int i = 1; i < to_sort.length() - 2; i++)
        {
            if (to_sort[i] > to_sort[i + 1])
            {
                changed = true;
                unsigned char tmp = to_sort[i];
                to_sort[i] = to_sort[i + 1];
                to_sort[i + 1] = tmp;
            }
        }
    }
    return to_sort;
}

// output the cost tree as a table to a file
void write_out_table(cost_tree_node* root)
{
    string output = "prefix,";
    for (int i = NODE_ZERO; i < NODE_ZERO + NUM_NODES; i++)
    {
        output += names[i - NODE_ZERO];
        output += ",";
    }
    output += "\n";

    queue<cost_tree_node*> row_queue;
    row_queue.push(root);

    int block = 0;
    while (!row_queue.empty())
    {
        cost_tree_node* row_starter = row_queue.front();
        row_queue.pop();

        if (row_starter->children == NULL) continue;

        if (row_starter->planets_sequence.length() - 1 > block)
        {
            for (int i = 0; i < NUM_NODES + 1; i++)
            {
                output += " ,";
            }
            output += "\n";
            block = row_starter->planets_sequence.length() - 1;
        }

        for (unsigned char c : row_starter->planets_sequence)
            output += toupper(names[c - NODE_ZERO][0]);
        output += ",";

        for (int i = 0; i < NUM_NODES; i++)
        {
            if (row_starter->children[i] == NULL)
            {
                output += "-,";
                continue;
            }
            output += to_string(row_starter->children[i]->cumulative_cost);
            output += ",";
            row_queue.push(row_starter->children[i]);
        }
        output += "\n";
    }

    ofstream file;
    file.open(names[root->planets_sequence[0] - NODE_ZERO] + ".csv");
    file << output;
    file.close();
}

// build the cost tree, this is the actual dynamic programming bit
cost_tree_node* build_dynamic_cost_tree(unsigned char start_node_index)
{
    // make the specified starting node be the root of the tree
    string root_sequence; root_sequence.push_back(start_node_index);
    cost_tree_node* root = new cost_tree_node
    {
        0,
        weight[start_node_index - NODE_ZERO],
        root_sequence,
        start_node_index,
        NULL,
        NULL
    };

    // nodes that need to have their children populated in this block
    queue<cost_tree_node*> this_block_nodes;

    // new child nodes which are the best route starting at string[0] and ending at string[-1]
    // i.e. these are the best (cheapest) permutations of a sequence of planets
    map<string, cost_tree_node*> next_block_routes;

    this_block_nodes.push(root);

    // repeat until we reach a block containing cells representing entire routes through the network
    for (int block = 0; block < NUM_NODES - 1; block++)
    {
        // populate all the rows in the current block
        while (!this_block_nodes.empty())
        {
            // populate the children of a node
            // the parent represents the row label on the left side of a table
            cost_tree_node* parent = this_block_nodes.front();
            this_block_nodes.pop();

            parent->children = new cost_tree_node * [NUM_NODES];

            // calculate the costs of each possible child node (table cell) from the current parent (table row)
            for (unsigned char c = NODE_ZERO; c < NUM_NODES + NODE_ZERO; c++)
            {
                if (parent->planets_sequence.find(c) != string::npos)
                {
                    // discard if the sequence has duplicate planets
                    parent->children[c - NODE_ZERO] = NULL;
                }
                else
                {
                    // create a new child node (table cell) and calculate its cumulative weight and cost
                    string node_sequence = parent->planets_sequence;
                    node_sequence += c;
                    cost_tree_node* node = new cost_tree_node
                    {
                        parent->cumulative_cost + (parent->cumulative_weight * adjacency[parent->last_planet - NODE_ZERO][c - NODE_ZERO]),
                        parent->cumulative_weight + weight[c - NODE_ZERO],
                        node_sequence,
                        c,
                        NULL,
                        parent
                    };
                    parent->children[c - NODE_ZERO] = node;
                    string sorted_seq = sort_sequence(node->planets_sequence);
                    if (block >= 2)
                    {
                        // check to see if this node represents the cheapest way to travel between
                        // its set of planets, with the same start and end points
                        auto current_best = next_block_routes.find(sorted_seq);
                        // if there are no other routes like this, it must be the best
                        if (current_best == next_block_routes.end()) next_block_routes.insert({ sorted_seq, node });
                        // if there are other routes and this one is the cheapest, update it as the cheapest
                        // so that it gets computed in the next block
                        else if (node->cumulative_cost < (*current_best).second->cumulative_cost) next_block_routes[sorted_seq] = node;
                        // otherwise discard it
                    }
                    else
                    {
                        // add the node to the map so that we will compute its children in the next block
                        next_block_routes.insert({ sorted_seq, node });
                    }
                }
            }

        }

        // queue up the best routes (table cells) from the last block for evaluation in the next one
        // where they now become the table rows
        for (pair<string, cost_tree_node*> pr : next_block_routes)
        {
            this_block_nodes.push(pr.second);
        }

        // clear and start again
        next_block_routes.clear();
    }

    // write the node tree out as a table to a file
    write_out_table(root);

    // finally iterate over the list of best routes (table cells) in the last block and find the cheapest one
    cost_tree_node* best_route_through_table = this_block_nodes.front();
    while (!this_block_nodes.empty())
    {
        cost_tree_node* front = this_block_nodes.front();
        this_block_nodes.pop();
        if (front->cumulative_cost < best_route_through_table->cumulative_cost)
        {
            best_route_through_table = front;
        }
    }

    // return the node describing the best (cheapest) way of traversing the graph, starting at the specified starting point
    return best_route_through_table;
}

int main()
{
    for (int i = NODE_ZERO; i < NUM_NODES + NODE_ZERO; i++)
    {
        cost_tree_node* res = build_dynamic_cost_tree(i);
        cout << res->cumulative_cost * 25 << endl;
        for (unsigned char c : res->planets_sequence) cout << names[c - NODE_ZERO] << " ";
        cout << endl << endl;
    }
}
Copy
\end{verbatim}

As can be seen from the console output of the code, by looking at the
\emph{lowest cost table cell} in the \emph{last block of each table}
(I\textquotesingle m defining a block as a set of rows which have the
same number of previously visited planets shown in the far left column,
so block 0 has \textquotesingle A\textquotesingle{} in the left column,
block 1 will have \textquotesingle AB\textquotesingle,
\textquotesingle AG\textquotesingle,
\textquotesingle AD\textquotesingle,
\textquotesingle AE\textquotesingle), we can find the cheapest route
starting at the origin node of the table:

\begin{itemize}
\tightlist
\item
  starting at alpha: 69750 (alpha -\textgreater{} delta -\textgreater{}
  epsilon -\textgreater{} beta -\textgreater{} gamma)
\item
  starting at beta: 105250 (beta -\textgreater{} epsilon -\textgreater{}
  delta -\textgreater{} alpha -\textgreater{} gamma)
\item
  starting at gamma: 12600 (gamma -\textgreater{} delta -\textgreater{}
  alpha -\textgreater{} beta -\textgreater{} epsilon)
\item
  starting at delta: 69000 (delta -\textgreater{} alpha -\textgreater{}
  epsilon -\textgreater{} beta -\textgreater{} gamma)
\item
  starting at epsilon: 69750 (epsilon -\textgreater{} delta
  -\textgreater{} alpha -\textgreater{} beta -\textgreater{} gamma)\\
  The \emph{best route overall} can be found by taking the cheapest of
  these optimal routes, DAEBG for 69000. This is the same optimal route
  found by brute force, as we would expect (in fact, we can verify that
  the optimal route costs starting from other planets are also the best
  routes found starting from those planets by looking at the results of
  the brute force method).
\end{itemize}

This dynamic approach is guaranteed to find the optimal route, because
we only prune routes which visit the \textbf{same planets} (and thus
have the same weight), and \textbf{end at the same planet} (i.e. have
the same options/edge costs for future traversal steps) but with a
\textbf{worse cost than other routes covering the same planets}.

In terms of complexity, it\textquotesingle s evident to see that this is
faster than the brute force approach, for two reasons, which correspond
to the two main techniques the dynamic approach uses:

\begin{enumerate}
\tightlist
\item
  Memoisation - each time we want to calculate the cost of traversing
  from one node to another, we don\textquotesingle t recalculate the
  entire cost, just the progressive cost, and previous calculations are
  saved and reused (reduces time cost to calculate multiple branching
  routes)
\item
  Pruning - by pruning provably inferior routes at early stages, we
  massively reduce the search space. in fact, we reduce our search space
  all the way down to just 60 full routes covered, from 120 before.
\end{enumerate}

Writing code for this allowed me to test with different numbers of
nodes:

\begin{longtable}[]{@{}lllll@{}}
\toprule\noalign{}
n & routes checked to completion & nodes evaluated & total possible
routes & nodes evaluated in brute force (equivalent) \\
\midrule\noalign{}
\endhead
\bottomrule\noalign{}
\endlastfoot
5 & 60 & 260 & 120 & 600 \\
6 & 120 & 990 & 720 & 4320 \\
7 & 210 & 3402 & 5040 & 35280 \\
8 & 336 & 10808 & 40320 & 322560 \\
9 & 504 & 32328 & 362880 & 3265920 \\
\end{longtable}

With this table we can see the huge benefit to pruning compared with the
brute force approach. The pattern formed is that the number of routes
checked to completion is {} when {}. This makes sense since at each
step, we prune such that the number of routes to examine in the next
block is halved, then thirded, etc, leaving only {} routes checked to
completion.

We can also see, with some calculation, that the number of actual
evaluations (i.e. calculating the cost of a node, and deciding if we
should prune it or carry it forward) is equal to {}; this represents the
total number of rows in the table multiplied by the number of filled
cells in each row, block by block (where {} is the index of the block).
This can be simplified to {}, and our pruning becomes even clearer, as
we\textquotesingle re multiplying the {} total number of routes by
summed fractions, where each fraction is representing 1 divided by the
ratio of nodes we can prune at each step in the table. the equivalent in
brute-force is just the number of routes multiplied by the number of
nodes to represent the time taken to calculate the cost of a particular
route ({} routes, each of length {}), so {}. Again we can see that our
algorithm is much, much better than brute force in terms of complexity.

We need to consider the complexity of the process of checking for
alternative routes with the same nodes
(\textquotesingle ABGD\textquotesingle{} vs
\textquotesingle AGBD\textquotesingle). My implementation uses a simple
{} bubble sort, so we can say that overall this implementation has a
time complexity of {}. The algorithm could be improved with the use of a
better method for detecting permutated sequences of planets (ABGD vs
AGBD) which doesn\textquotesingle t use sorting but instead hashes the
sequence (which could be done in linear {} )time.

\subsection{Task 5}\label{task-5}

The art gallery problem is a geometric problem in which an uneven,
concave polygon (i.e. 2D shape, though the problem also exists for 3D
polyhedra but is much harder to solve) must have the minimum possible
number of \textquotesingle guards\textquotesingle{} posted at discrete
points on or within the polygon such that the entire polygon is
\textquotesingle visible\textquotesingle{} to the guards (i.e. there is
an unbroken ray that leads from any point on any edge to at least one
guard). The problem specifically is finding the minimum number of
guards.

The analogy is referential to an art gallery of course, with various
rooms of different shapes, which is likely to be concave and possible
have disconnected obstacles within the gallery. In this scenario, we
need to be able to see the whole gallery at once to keep the artworks
safe from theft or vandalism, while hopefully minimising the number of
guards required to guard it at once. We assume that each guard has 360
degree vision.

Václav Chvátal showed that the maximum possible number of guards
required was equal to {} , where {} is the number of vertices in the
polygon. It can be seen that a single guard must be able to observe the
whole of a convex shape (of which a triangle is the simplest and always
convex), since no matter where within or on a triangle we place our
observation point, we can draw direct lines to all the corners of the
triangle. We can see therefore that Chvátal\textquotesingle s theorem is
true, because if we triangulate the polygon (i.e. we ensure the entire
shape is made up of shapes of only 3 vertices) the maximum possible
number of triangles we could end up with is {} , for the case where the
gallery is made up of a number of disconnected triangles which share no
vertices with one another, and since we need exactly one guard per
triangle, we can then see that the maximum possible number of guards
needed would be also {} \textsuperscript{{[}1{]}}.

However, our number of guards is reduced when we consider that many
triangles will share at least one vertex with a neighbour, usually
sharing two with any particular neighbour, which reduces the number of
guards required, saving one guard each time this happens. The difficulty
of the art gallery problem lies in decomposing a complex shape with many
\textquotesingle ins and outs\textquotesingle{} and placing guards
optimally.

Continuing this train of thought, it is true that we can triangulate our
polygon and then colour each of its vertices one of three colours, such
that all triangles have one of each of the three colours on
it\textquotesingle s vertices. Steve Fisk points out that by taking the
total number of vertices of a certain colour, specifically the colour
with the fewest instances in the polygon (i.e. in a polygon with 2 red,
1 green and 1 blue vertices, we would take either 1 green or 1 blue) we
can reduce our maximum number of guards
required\textsuperscript{{[}2{]}}. This is essentially a mathematical
form of the \textquotesingle sharing vertices\textquotesingle{} concept
described in the previous paragraph.

Both of these geometric proofs are useful for reducing the search space
in terms of finding solutions for smaller numbers of guards by setting
an upper bound. However, these approaches are somewhat naive as they
cannot optimise concave shapes where vertices are not shared. See the
diagram below:\\

\includegraphics{blob:app://obsidian.md/09d5cb7a-3164-42f0-a4b2-edf61c2ce313}

\hfill\break
Chvatal\textquotesingle s proof would tell us that we need a maximum of
three guards (since we have seven total vertices, and we have to round
up), and Fisk\textquotesingle s proof and the colouring scheme tells us
that we need at most two guards: these could be placed at the two green
vertices, or the two blue ones. However, looking from above, we can
clearly see that we only need to place a single guard at the highlighted
green vertex. Everything that the other green vertex can see, can also
be seen by the highlighted vertex, plus a bit more.

An algorithm to optimise this problem to minimise the number of guards
would need to be able to look at different combinations of guard
placements to see if the number of guards can be reduced (i.e.
brute-force). Heuristics could be applied for example by counting around
vertices and looking at their angles relative to the origin vertex to
see if there are occluded (invisible from that point) areas.
Approximation methods might use a grid to check the coverage of the
polygon from certain vertices in the shape.

It\textquotesingle s important to note that we have an additional
constraint in this problem: keeping guards on vertices. However, there
are variations of the problem (and indeed, real-world applications light
lighting a stage would be less constrained) which allow guards to be
placed on edges, or even anywhere within the polygon, massively
(ordinally) increasing the search space, by way of the number of
possible configurations.

One approach presented by Ghosh is to reduce the the overall polygon to
a set of convex polygons, each of which may be observed by a single
guard\textsuperscript{{[}3{]}}. However, even this may not produce
optimal results, see the diagram above once again.

The problem, depending on constraints, is considered NP-hard, meaning it
it\textquotesingle s both difficult to solve and difficult to verify in
polynomial (i.e. {}) time.

\begin{center}\rule{0.5\linewidth}{0.5pt}\end{center}

\begin{enumerate}
\tightlist
\item
  \phantomsection\label{fn-1-aa358eb53a96208e}{Chvátal, V. (2004)
  \emph{\textquotesingle A combinatorial theorem in plane
  geometry\textquotesingle{}}, \emph{Journal of Combinatorial Theory,
  Series B}. Available at:
  \url{https://www.sciencedirect.com/science/article/pii/0095895675900611}
  (Accessed: 29 October 2023).}
\item
  \phantomsection\label{fn-2-aa358eb53a96208e}{Aigner, M., Ziegler, G.M.
  (2018). \emph{\textquotesingle How to guard a museum. In: Proofs from
  THE BOOK\textquotesingle{}}. Springer, Berlin, Heidelberg.
  \url{https://doi.org/10.1007/978-3-662-57265-8_40}}
\item
  \phantomsection\label{fn-3-aa358eb53a96208e}{Ghosh, S. K. (1987),
  \emph{\textquotesingle Approximation algorithms for art gallery
  problems\textquotesingle{}},~\emph{Proc. Canadian Information
  Processing Society Congress}, pp.~429--434.}
\end{enumerate}

\end{document}
