% Options for packages loaded elsewhere
\PassOptionsToPackage{unicode}{hyperref}
\PassOptionsToPackage{hyphens}{url}
%
\documentclass[
]{article}
\usepackage{amsmath,amssymb}
\usepackage{iftex}
\ifPDFTeX
  \usepackage[T1]{fontenc}
  \usepackage[utf8]{inputenc}
  \usepackage{textcomp} % provide euro and other symbols
\else % if luatex or xetex
  \usepackage{unicode-math} % this also loads fontspec
  \defaultfontfeatures{Scale=MatchLowercase}
  \defaultfontfeatures[\rmfamily]{Ligatures=TeX,Scale=1}
\fi
\usepackage{lmodern}
\ifPDFTeX\else
  % xetex/luatex font selection
\fi
% Use upquote if available, for straight quotes in verbatim environments
\IfFileExists{upquote.sty}{\usepackage{upquote}}{}
\IfFileExists{microtype.sty}{% use microtype if available
  \usepackage[]{microtype}
  \UseMicrotypeSet[protrusion]{basicmath} % disable protrusion for tt fonts
}{}
\makeatletter
\@ifundefined{KOMAClassName}{% if non-KOMA class
  \IfFileExists{parskip.sty}{%
    \usepackage{parskip}
  }{% else
    \setlength{\parindent}{0pt}
    \setlength{\parskip}{6pt plus 2pt minus 1pt}}
}{% if KOMA class
  \KOMAoptions{parskip=half}}
\makeatother
\usepackage{xcolor}
\usepackage[margin=2cm]{geometry}
\setlength{\emergencystretch}{3em} % prevent overfull lines
\providecommand{\tightlist}{%
  \setlength{\itemsep}{0pt}\setlength{\parskip}{0pt}}
\setcounter{secnumdepth}{-\maxdimen} % remove section numbering
\ifLuaTeX
  \usepackage{selnolig}  % disable illegal ligatures
\fi
\IfFileExists{bookmark.sty}{\usepackage{bookmark}}{\usepackage{hyperref}}
\IfFileExists{xurl.sty}{\usepackage{xurl}}{} % add URL line breaks if available
\urlstyle{same}
\hypersetup{
  pdftitle={Literature Review},
  hidelinks,
  pdfcreator={LaTeX via pandoc}}

\title{Multiplayer Games Programming Literature Review - Jacob Costen
(23025180)}
\author{}
\date{}

\begin{document}
\maketitle

\hypertarget{introduction}{%
\subsection{Introduction}\label{introduction}}

Developing networked games introduces significant challenges, from
management of game state within the game client, to synchronisation of
the game state between clients and servers. Multiplayer programming
requires heavy use of concurrent programming which comes with its own
complexities, and different game genres often present different
requirements which developers must accommodate when designing networking
code (simulation games such as Factorio present different challenges to
first-person shooters such as Call of Duty). This review will consider
some of the problems and techniques that arise both with multi-threaded
programming, and networked multiplayer game programming.

\hypertarget{literature-review}{%
\subsection{Literature Review}\label{literature-review}}

Networked games require the use of concurrency, since network events
happen asynchronously to game events and updates. Concurrent code
introduces new programming challenges which arise from multiple
execution threads accessing shared resources simultaneously.

Race conditions occur when multiple threads try to access and modify
data simultaneously (Netzer and Miller (1992))\textsuperscript{{[}1{]}}.
For instance, where two threads attempt to increment a counter
\texttt{i} simultaneously, \texttt{i} is read by thread A, then by
thread B, then each thread adds one and writes the result back to
\texttt{i}. The counter only increased by one, because the read
operation on thread B occurred before the write operation on thread A.
When the two threads are almost synchronised with one another, this race
condition becomes apparent and one of the increments is missed
(potentially causing further problems later in the code).

The simplest synchronisation structure which is used to prevent race
conditions is a mutex. Mutices (a contraction of \textquotesingle mutual
exclusion\textquotesingle) may be
\textquotesingle locked\textquotesingle{} by one thread, and any other
attempt to lock that mutex will block until that mutex is released by
the original thread. In C\# mutices can be constructed automatically
using the \texttt{lock(object)\ \{\ \}} syntax, and will be released
when execution exits the code contained in the lock block (Agafonov
(2016))\textsuperscript{{[}2{]}}. However, according to Voss, Asenjo,
and Reinders (2019)\textsuperscript{{[}3{]}}, the use of mutices and
other synchronisation structures should be minimised when building
high-performance code, since they often introduce bottlenecks where
concurrent threads waste time waiting for resources to become available.
Instead, algorithms should be restructured to eliminate the need for
these structures.

The use of locking structures as described above introduces a problem called deadlocking. This occurs when thread A locks resource A, and waits for thread B to release resource B in order to lock it, while thread B is simultaneously waiting for thread A to release resource A before it will release resource B. The result is a deadlock, where both threads wait on each other endlessly. Zhou, et al. (2017)\textsuperscript{{[}4{]}} develop a method for detecting potential for deadlocks based on the key property that locks must be nested (one resource held while waiting for another) in order to generate deadlock conditions; however they also note that detecting deadlocks reliably is very difficult, especially using standard testing, due to the fact that both threads need to acquire their first resource before either tries to acquire their second resource. The authors further conclude that the best way to resolve deadlocks like this without significantly redesigning the software is to merge multiple timing-sensitive fine-grained locks in to a single coarser lock: meaning that when thread A locks resource A, it is reserving access to resource B at the same time, through a single synchronisation object (e.g. a mutex).

When creating a networked game, developers must choose a transport layer
protocol over which to communicate gameplay data. The two predominantly
used protocols are TCP (transport control protocol) and UDP (user
datagram protocol). While older games - such as World of Warcraft, as
studied by Wang, et al. (2012)\textsuperscript{{[}5{]}} - tend make use
of TCP, many other titles use UDP instead. The key difference between
the two protocols is that TCP guarantees the arrival, integrity, and
order of packets as inherent functionality, while UDP leaves the
implementation of this up to the developer. This means that while TCP
might be more reliable, it is often slower, requiring transport-level
acknowledgement packets to be sent back and forth (Glazer and Madhav
(2016))\textsuperscript{{[}6{]}}. Thus, the use of UDP is preferred, as
it gives more control over exactly how much reliability is prioritised
over speed.

While a combined approach may initially appear ideal for a game where
different packets have varying priority, Saldana
(2015)\textsuperscript{{[}7{]}} finds that the use of TCP and UDP
together actually harms network performance, since TCP connections tend
to fill a connection\textquotesingle s bandwidth to the point that the
time taken for multiplexing those connections with background UDP
traffic leads to poorer network performance and increased delay in the
game.

A challenge faced by online multiplayer games is in realtime replication
of game worlds. Specifically, when a gameplay element changes on one
client (for instance, a player moves in the world), that information
must be sent to other clients in order to correctly replicate that
movement. However, periodically updating the remote
player\textquotesingle s position results in jerky movement, so
prediction techniques must be employed to move the remote player
continuously, even though position packets only arrive periodically.

One of these techniques is dead reckoning, which estimates an objects
position based on past motion or velocity data and the time since the
last position update (Shi, Corriveau, and Agar
(2014))\textsuperscript{{[}8{]}}. This minimises the jump in position
when the next update arrives. Dead reckoning also has the advantage that
it does not require network packets to arrive any time the player
changes direction or velocity, minimising the effects of packet loss
over the network.

More precise algorithms may be applied depending on the specific use
case; games like snake maintain continuous movement and so simple
continuous-motion prediction may be used. Conversely, many gameplay
environments (particularly first-person shooters) require more advanced
algorithms to accurately predict player movement, such as the
\textquotesingle EKB\textquotesingle{} algorithm proposed by Shi,
Corriveau, and Agar (2014)\textsuperscript{{[}8{]}}, or the
neural-network-based approach taken for predicting high-speed vehicle
motion in multiplayer games by Walker, et al.
(2024)\textsuperscript{{[}9{]}}. Both of these examples illustrate that
the best algorithm is often one which is developed specifically for the
game it is to be used in; this is something machine learning offers
potential with due to its ability to be trained on existing data.

Maintaining fairness in multiplayer games is one of the largest
challenges faced by developers. This manifests in two key ways: cheating
and lag. Players who experience greater lag (usually measured as higher
ping, the time required for a packet to make a round trip between the
client and server) are significantly disadvantaged compared to other
players. Geographical distance to the server, quality of network
hardware, and the type of physical link (both inside and outside the
player\textquotesingle s home network) contribute to this, and this is
often felt disproportionately by poorer players. One way game developers
can combat this is by normalising player lag to prevent players on
faster connections having a significant advantage. Zander, Leeder, and
Armitage (2005)\textsuperscript{{[}10{]}} developed a tool which is able
to reside between the game server and the internet to implement this
behaviour.

Cheating is a wide field of problems in the multiplayer games industry,
and these are mitigated in a variety of ways. Lehtonen
(2020)\textsuperscript{{[}11{]}} categorises cheating into soft (making
use of bugs or exploits to gain an advantage) and hard (making
modifications to the game client or using additional tools) cheats, and
notes that preventing hard cheats is the objective of most anti-cheat
systems. Often soft cheats are fixed by developers releasing game
patches to remove exploits.

Anti-cheat software may be applied either server-side or client-side
(often both): server anti-cheat operates on the principle that the
client should never be trusted; every action and packet should be
validated before it is accepted (for instance, many Minecraft
multiplayer servers feature anti-cheat tools which will disconnect a
player if they have been airborne for too long and are deemed to be
\textquotesingle flying\textquotesingle, a common form of cheating).
Server-side anti-cheat may also involve analysis of activity logs to
detect suspicious patterns, such as the use of bots, which may produce
excessively regular or perfect movement, aim, or activity.

Client-side anti-cheat usually involves having a side-process which
monitors the state of the game client to detect changes to its memory
contents. For instance, an anti-cheat system may detect when a cheater
attempts to inject new code into the game client, and report this
activity to the server (Lehtonen (2020))\textsuperscript{{[}11{]}}.
However, this often introduces serious security problems, since in order
to do analyse the game client\textquotesingle s memory, the anti-cheat
tool must run at the kernel level of the host system. Poorly secured,
unstable, or resource-intensive kernel-level anti-cheat systems may
cause harm to players\textquotesingle{} machines, and provide a vast
attack surface for hackers via the \textquotesingle Bring Your Own
Vulnerable Driver\textquotesingle{} method for gaining access to a
system; this is of particular concern since these anti-cheat tools
interface with the network constantly (Basque-Rice
(2023))\textsuperscript{{[}12{]}}.

When dealing with cheaters, live service game providers such as Blizzard
and Riot Games will often ban cheaters in waves, rather than
individually as they are detected. This has a number of advantages:
dramatic monthly ban numbers have a greater psychological deterrent
effect for potential cheaters; cheat developers may be financially
damaged by an influx of refunds from players who bought cheating tools
from them; and cheat developers have a much harder time identifying why
or when their tools were detected by anti-cheat systems, slowing down
the development of more advanced, harder to detect, cheats (Laato, et
al. (2021))\textsuperscript{{[}13{]}}.

\hypertarget{conclusion}{%
\subsection{Conclusion}\label{conclusion}}

The state of multiplayer gaming has changed significantly over the last
two decades, following improvements in network infrastructure,
development of standardised libraries and protocols, and changes in
studio priorities and consumer habits. In particular, intrusive
anti-cheat systems have become increasingly common.\\
While many of the underlying challenges involved with synchronising
dozens of players from geographically different places to within
fractions of a second of one another remain largely the same,
machine-learning-powered prediction algorithms present an interesting
avenue for improving in-game experience for players.

\hypertarget{bibliography}{%
\subsection{Bibliography}\label{bibliography}}

{[}1{]}: Netzer, R.H. and Miller, B.P. (1992) `What are race
conditions?', \emph{ACM Letters on Programming Languages and Systems},
1(1), pp. 74--88. doi:10.1145/130616.130623.\\
{[}2{]}: Agafonov, E. (2016) \textquotesingle Multithreading with C\#
Cookbook\textquotesingle. \emph{Packt Publishing Ltd}.\\
{[}3{]}: Voss, M., Asenjo, R. and Reinders, J. (2019) `Synchronization:
Why and how to avoid it', \emph{Pro TBB}, pp. 137--178.
doi:10.1007/978-1-4842-4398-5\_5.
{[}4{]}: Zhou, J. \emph{et al.} (2017) `Undead: Detecting and preventing deadlocks in production software', \emph{2017 32nd IEEE/ACM International Conference on Automated Software Engineering (ASE)}, pp. 729-740. doi:10.1109/ase.2017.8115684.\\
{[}5{]}: Wang, X. \emph{et al.} (2012) `Characterizing the gaming
traffic of World of Warcraft: From game scenarios to Network Access
Technologies', \emph{IEEE Network}, 26(1), pp. 27--34.
doi:10.1109/mnet.2012.6135853.\\
{[}6{]}: Glazer, J.L. and Madhav, S. (2016) \emph{Multiplayer Game
Programming: Architecting Networked Games}. New York: Addison-Wesley.\\
{[}7{]}: Saldana, J. (2015) `On the effectiveness of an optimization
method for the traffic of TCP-based multiplayer online games',
\emph{Multimedia Tools and Applications}, 75(24), pp. 17333--17374.
doi:10.1007/s11042-015-3001-y.\\
{[}8{]}: Shi, W., Corriveau, J.-P. and Agar, J. (2014) `Dead reckoning
using play patterns in a simple 2D multiplayer online game',
\emph{International Journal of Computer Games Technology}, 2014, pp.
1--18. doi:10.1155/2014/138596.\\
{[}9{]}: Walker, T. \emph{et al.} (2024) `Predictive dead reckoning for
online peer-to-peer games', \emph{IEEE Transactions on Games}, 16(1),
pp. 173--184. doi:10.1109/tg.2023.3237943.\\
{[}10{]}: Zander, S., Leeder, I. and Armitage, G. (2005) `Achieving
fairness in multiplayer network games through automated latency
balancing', \emph{Proceedings of the 2005 ACM SIGCHI International
Conference on Advances in computer entertainment technology}, pp.
117--124. doi:10.1145/1178477.1178493.\\
{[}11{]}: Lehtonen, S. (2020) \textquotesingle Comparative study of
anti-cheat methods in video games\textquotesingle,~\emph{University of
Helsinki, Faculty of Science}.\\
{[}12{]}: Basque-Rice, I. (2023)~\textquotesingle Cheaters Could Prosper:
An Analysis of the Security of Video Game
Anti-Cheat\textquotesingle,~\emph{Doctoral dissertation, Abertay
University}.\\
{[}13{]}: Laato, S. \emph{et al.} (2021) `Technical cheating prevention
in location-based games', \emph{International Conference on Computer
Systems and Technologies '21}, pp. 40--48.
doi:10.1145/3472410.3472449.\\

\end{document}
